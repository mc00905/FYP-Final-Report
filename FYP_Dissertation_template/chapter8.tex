\chapter{Conclusion}
\section{Overview of the Project}

We started the project by looking at the lessons learned during my time at Hindsight, and identified a number of common problems that have been observed when working with a RESTful Micro-service architecture. These issues were used as the inspiration for the project and were used to produce both the Objectives for the project as a whole, and the Requirements for the actual framework produced. We looked at best practices and standards for RESTful APIs in general alongside a couple of contemporary alternative API methodologies. We also looked at what a micro-service is at its core, and how the approach differs from a more traditional monolithic software architecture.

Ethical concerns for the project were then detailed and the actual Requirements for the software components were defined. Once I had the list of Requirements, work began on the design of the framework itself, both on the development on the code, and the process for ensuring it was tested properly.

Overall the project has been a success. As shown in the latter half of Chapter 6 and in Chapter 7 the Requirements for the different software components were all met, and the actual framework produced works well. 

The framework provides the facilities to quickly develop RESTful micro-services and allows generation of an API client from the services API specification. The micro-services serve API documentation from a documentation endpoint providing details for entire API. This includes all the information needed to effectively make use of the API, whether it be information about how requests are to be made to the API or the shape and format of the responses returned by it.

A unit test-suite was developed for each of the examples showcasing the framework, covering all of the cases laid out by the Software Requirements. In Chapter 6 we showed that this test-suite not only passed fully, but also provided ample code coverage of the software. This level of code coverage allows more confidence to be had in the tests, as it shows exactly what is and isn't covered.

The biggest indicator of success though is the fact that the framework is already starting to see usage at Hindsight, showing that it is in a good enough state to produce production level code used in our live services.

\section{Enhancing the Framework in the Future}
As we increase the number of micro-services at Hindsight, and as we look to improve things like scalability of the individual services, additional features and refinements to existing features are going to be necessary. The framework and client generation is pretty dynamic, with changes easily injected into the existing code, allowing the software to be molded as we need it to. While nothing is essential at the time of writing, below I have detailed some potential additions and changes that could be made that fell out of the scope of the project.
\subsection{More advanced Rate-limit support for the micro-services}
The express-rate-limit library used in the exemplar micro-service is handy, but relatively simple. As a proof of concept for handling rate limiting it works fine, and for general usage of a small scale web application it's functional. However, it pools all of the individual users of the service under one roof, and applies a global rate-limit to everyone at the same time.

The primary application we work on at Hindsight embeds itself into Jira, and so deals with both a large number of individual users, but also a large set of tenant, the tenant being the organisation that the individual user falls under. In our use-case, it makes more sense to have a tenant based rate-limit.

Typically this would be done through use of an in-memory database such as Redis to create a data entry tracking the requests each tenant makes, with each request incrementing a counter value. Once the counter reaches the limit, a flag will be tripped and will block subsequent requests with a rate-limit network message. A timestamp enforcing the end of the limit will be created, and once that time has been reached, the limit will be reset and we will start again.
\subsection{More intelligent mechanism for retrying rate-limited requests}
Like with the rate-limiting of the micro-service, the automatic retry mechanism is functional, but fairly simple. The axios-retry library uses a set of user-defined status code to check whether the failed request should be retried. When a request fails, a HTTP status code is returned to the client, and the library compares this against the user defined list and if it matches retries it. 

This works for most failed requests, but specific handling in the case of rate-limited requests would make it a bit more robust. By pulling out the expiration timestamp for the current rate-limit we could potentially retry a limited request at a point in the future where we know it will be accepted. However, the user doesn't want to be waiting a long time for the request to be retried, so logic for determining an acceptable time-frame would have to be added too, which turns an otherwise simple addition into a more difficult task.
