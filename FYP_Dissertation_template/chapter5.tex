\chapter{System Design}
\section{Introduction}
In this section of the report I will explain and justify the design decisions I have made relating to both the architecture of my systems, and the libraries they employ.

How is everything broken down. Rough outline for each of the sections and why I have made the decisions to do it this way

API -> API Client Diagram

How is everything gonna gel together at the end

Big picture - blue sky aim 

\section{Technologies and Libraries}

What are they, why them, brief discussion about competitors etc, how do they fit in/what are they doing for me specifically

    \subsection{Node.js}
    \subsection{TypeScript}
    \subsection{OpenAPI Specification}
    \subsection{TSOA}
    \subsection{Express}
    \subsection{Swagger-UI-Express}
    \subsection{Express-Rate-Limit}
    \subsection{Mongoose}
    \subsection{Axios}
    \subsection{Axios-Retry}
    \subsection{Jest}
    \subsection{Nock}
    \subsection{OpenAPI Code Generator}
    
\section{System Architecture}
\subsection{Microservice}
Micro-service layer diagrams -> flow chart showing data flow - explaination of each layer

Discussion/diagrams of interweaving of micro-service libraries/modules?

Discussion on code implementations -> typical jargon and talk about decisions made on fly, any tweaks/issues

Focus on TSOA and the TSOA Spec command for generating an OpenAPI spec file from the controllers -> potentially for CI pipelines and long term goal of using the spec as a client gen -> publish as module for import elsewhere

The /documentation endpoint for providing living documentation from the same spec file

\subsection{API Client}
As above

Additional information about designing it as an NPM module, dependency that can be consumed 

Can I talk about it being published without actually having to publish the package? No big deal if so, just avoids some hassle.

\subsection{Modifying the OpenAPI Code Generator}

What does it do - what needs changing? Kinda cheating but there's 5/6 potential choices for TypeScript frameworks for the generated code to use, there's an Axios option available so ties in pretty perfectly with my handcrafted-client, I guess it's worth evaluating some of the other options though? For example the TypeScript-Node option uses the hugely popular but now deprecated Request library so can talk about longetivity of the generated code from that regard?

What functionality is missing from the auto-gen that my client has

Venn diagram a good way of showing this? Any other visual options you can think of?

How am I gonna implement it (for the most part will essentially copy and paste into the template files) -> talk about best way to add it to mustache template files - although for the most part the additions will be static imports/code additions, no need for pulling templating fields out



\section{API Design}
Why are the API endpoints the way they are
- nouns
- plurals
- talk about ensuring usage of a case for each HTTP verb is used GET,PUT,POST,DELETE
- JSON format -> standard and friendly
- Rate limiting and pagination on Array results