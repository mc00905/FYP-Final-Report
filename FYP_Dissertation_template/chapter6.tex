\chapter{Testing and Evaluation}
\section{Introduction}

This section will detail my plans on how I will test my solutions, and how I will determine how they have measured up against the specifications defined for them.

There will be two main testing approaches throughout the project, individual Unit Testing on different components of the software, and more end-user oriented end-to-end testing where I will be actual using the functionality and deeming it functional. 
\section{Unit Testing with SuperTest, Jest and Nock}

My Unit Tests for both the micro-service and the client will use a number of key libraries, SuperTest, Jest and Nock. Whilst they will test the sample services I have created, the main aim will be to produce a suite of tools for general micro-service testing.
\subsection{SupertTest}
SuperTest is a Node.js library used for testing HTTP. By passing in the server export from the App.ts we can perform Unit Tests on the actual HTTP requests made through the server.

All aspects of the response can be tested including headers and status code, with the body of the response being the focal point of my testing. This will allow me to ensure that everything returned by the API is returned in both the format and shape that is expected, and the appropriate errors are thrown on respective requests.
\subsection{Jest}
Jest is a lightweight and very user-friendly JavaScript testing framework developed by Facebook. The tests themselves are split into separate processes and run in parallel, resulting in extremely fast and performant test suites.

Jest requires very little boilerplate configuration, and pretty much works out of the box making it very easy to get straight to writing Unit Tests on the code-base.

The framework also has the Istanbul code coverage tool built in. Istanbul tracks how much of our code is actually covered by our Unit Tests such as:
\begin{itemize}
    \item Functions: Are all the functions and methods tested
    \item Branching: Are all branches in the decision-making logic tested
    \item Statements: Are all variable assignments, return values etc tested
\end{itemize}

The reports generated by Istanbul detail what code isn't covered too, allowing the test suite to be expanded to cover these areas. The high the code coverage, the more of the code-base that is tested, which in turn should reduce the numbers of bugs introduced to the system.

It should be noted that Code Coverage and Test Coverage are different. Where Code Coverage tests how much of the code is tested in some way, Test Coverage measures those tests against actual Specifications and Requirements, testing the actual coverage of the expected functionality. As such where Code Coverage shines for Unit Tests, Test Coverage is a very good metric for Acceptance Testing.

For this project I will only be testing Code Coverage as the bulk of my physical tests will be Unit Tests.
\subsection{Nock}
Nock is a Node.js library used for mocking HTTP requests. Nock intercepts the HTTP requests sent by the server or application, and rather than actually forwarding the error to the destination it returns a static response body or error defined by the developer. 

It is particularly useful for interacting with external APIs as it mimics the request sent by the consuming service without actually interacting within the API. This allows the unit tests to be run without connecting to the internet or running any of the dependencies of the service being tested locally. 

Nock allows creation of highly configurable mocks and every component of the requests and responses can be tweaked. This includes everything from headers to status codes.

Once a request is intercepted the mock is consumed. This allows different instances of the exact same request to return different responses. This is particularly useful for testing things like rate limit handling, as the first few requests which are limited will fail with a 429 status code, but subsequent requests that are not limited will return successfully.

All of the unit tests for my API Client will make use of Nock so that it can be tested in isolation from the micro-service.
\subsection{Unit Testing the Micro-service}
The testing of the micro-service will be broken down into two key sections, testing that the controllers work and respond as expected, and that the data-agents process the data correctly.
\subsubsection{Testing the Controllers}
In order to test the controllers and the REST API exposed through them I will make use of the SuperTest library. SuperTest allows programmatic HTTP requests to be made to the server within the unit tests, allowing testing of the same URL endpoints that the live API will expose.

Upon calling the HTTP requests I will use Jest assertions to verify that the fields returned within the response are correct. To do this I will have to create static response bodies for each of the requests that I wish to test and compare them against those actually returned by the request. If the fields match those expected, then the API is working as expected.

Each of the URL endpoints within the micro-service will have a small suite of unit tests written for them. Requests that should trigger successful responses will be validated as will as requests that should trigger expected errors, such as doing GET requests on resources that do not exist. The HTTP status codes will be tested, ensuring that they are being assigned correctly and any response or error bodies will also be checked to ensure that they contain all of the content that they should. Essentially any documented request and response for the API should be tested.
\subsubsection{Testing the Data Agents \& Models}
Testing the data-agents and models involves interacting with the actual database operations. We do not want our unit tests ever interacting with live data and so standard practise is to connect to a completely separate test-focussed database. This is easy to do with Mongoose, and we can simple create a new database connection as part of the unit tests. Once the database connection is defined we can simple inject a number of test documents into the collections to act as our test data.

A small sample of test documents will be created upon the start up of the unit tests and saved into the test database. Different values for each of the model fields will be used across the suite of documents. This will allow tests to be written that verify that the correct subset of the collection is returned for the different filtering and sorting options on the data-agent methods.

Each of the operation involved in fetching existing documents will utilise the test documents created upon database startup. This will allow assertions to be run against the actual document definitions reducing boilerplate and keeping the unit tests lightweight. However, for the operations involving modifying or creating documents the tests will consist of two stages. The first will be defining the parameters used for the creation or modification or the resource, as well as creating a JSON representation of the resource after the operation has been defined. The method can then be called assertions can be tested to ensure that the operation did not fail. In order to test that the actual data saved is what is expected, a second request fetching the new data needs to be made. This can be done directly through the mongoose model using the primary key used in the create or modify operation, and the response can be tested against the previously defined JSON object. This two-stage process ensures that not only does the operation not fail, the actual information stored in database is exactly what is expected.
\subsection{Unit Testing the Handcrafted Client}
\subsubsection{Intercepting HTTP Requests to the micro-service with Nock}
How do nock + jest tie together
testing each method - does it pull out the responses, handle Errror and rate limiting
\section{End-to-End Testing the Auto-generated Client}
\subsection{Calling the Micro-service Endpoints}
Actually press the buttons, use ts-node to call an index file with function calls. 
Does the generated code work - call endpoints correctly. Does typescript pick up the typings. Do the added features work
\section{Back to the Objectives}
\subsection{The Micro-service}
\subsection{The Handcrafted Client}
\subsection{The Auto-generated Client}
Have they been fulfilled etc
Why, why not